\documentclass[12pt]{article}
\usepackage[a4paper,top=20mm,bottom=20mm,left=20mm,right=20mm]{geometry}
\usepackage{url}
\usepackage{alltt}
\usepackage{xspace}
\usepackage{times}
\usepackage{listings}
\usepackage{bbm}
\usepackage{verbatim}
\usepackage{prognames}
\usepackage{optionman}
\usepackage{skaff}
\newcommand{\GRM}[0]{\texttt{grm}\xspace}
\newcommand{\Mup}[1]{\mathit{mup(s,#1)}}
\newcommand{\MS}[1]{\mathit{ms(s,#1)}}
\newcommand{\Lmin}[0]{\mathit{mup(s)}}
\newcommand{\Substring}[3]{#1[#2..#3]}
\newcommand{\EXECUTE}[1]{#1}

\title{Extracting unique substrings\\
a manual}
\author{\begin{tabular}{c}
         \textit{Stefan Kurtz}\\
         Center for Bioinformatics,\\
         University of Hamburg
        \end{tabular}}

\begin{document}
\maketitle

\section{The program \GRM}

The program \GRM is called as follows:
\par
\noindent\GRM [\textit{options}] \Showoption{query} \Showoptionkey{queryfile}
\par
The given \textit{queryfile} must be in multiple 
\Fasta format and can optionally be gzipped, in which case it must end 
with the suffix \texttt{gz}.
Each sequence in \texttt{queryfile} is called \textit{unit} in the following.
In addition to the mandatory option \Showoption{query}, the program
must be called with other option \Showoption{pck} or \Showoption{esa}
which specify to use a packed index or an enhanced suffix array for 
a given set of subject sequences.

The program runs in basically two modes. 

\begin{enumerate}
\item
In the default mode \GRM computes
for all positions \(i\) in each unit, say \(s\) of length
\(n\), the length \(\Mup{i}\) of the minimum unique prefix 
at position \(i\), if it exists. Uniqueness always refers to all substrings
represented by the index. \(\Mup{i}\) is defined by the following two 
statements:
\begin{itemize}
\item
If \(\Substring{s}{i}{n-1}\) is not unique in the index, then \(\Mup{i}=\bot\).
That is, it is undefined.
\item
If \(\Substring{s}{i}{n-1}\) is unique in the index, then \(\Mup{i}=m\), where 
\(m\) is the smallest value such that \(i+m-1\leq n-1\) and 
\(\Substring{s}{i}{i+m-1}\) occurs exactly once as a substring in the index.
\end{itemize}
If no option is used, then for each unit \(s\), \GRM computes 
\[\Lmin=\min\{\Mup{i}\mid i\in[0,n-1],\Mup{i}\neq\bot\}\]
It reports \(\Lmin\) and all positions \(i\) in unit \(s\)
satisfying \(\Mup{i}=\Lmin\). For each such \(i\), it also 
shows the corresponding unique absolute position in the index.
Note that it is possible that for all \(i\in[0,n-1]\) we have 
\(\Mup{i}=\bot\), which means that unit \(s\) does not contain any unique 
substring. In this case, \(\Lmin=\min\emptyset=\bot\), i.e.\ 
\(\Lmin\) is undefined and the program reports that for this unit there 
are no unique substrings.
\item
In the other mode, i.e.\ when option \Showoption{-ms} is used, the 
\textit{matching statistics} is computed. That is, for each position \(i\) in 
each unit, say \(s\) of length \(n\), \(\MS{i}=(l,j)\) is computed. Here
\(l\) is maximal such that \(\Substring{s}{i}{i+l-1}\) matches
a substring represented by the index and \(j\) is a start position of the
matching substring in the index. We say that \(l\) is the length of \(\MS{i}\)
and \(j\) is the subject position of \(\MS{i}\).
\end{enumerate}

The following options are available in \GRM:

\begin{Justshowoptions}
\Option{ms}{}{
Compute matching statistics of the units in the query file against the
given index.
}

\Option{fmi}{$\Showoptionarg{indexname}$}{
Use the old implementation of the FMindex. This option is not recommended.
}

\Option{esa}{$\Showoptionarg{indexname}$}{
Use the given enhanced suffix array to compute the matches.
}

\Option{pck}{$\Showoptionarg{indexname}$}{
Use the packed index (an efficient representation of the FMindex)
to compute the matches.
}

\Option{min}{$\ell$}{
Specify the minimum length $\ell$ of minimum unique prefixes or matching
statistics lengths to be reported. 
That is, in the default case,
for each unit \(s\) and each positions \(i\) in \(s\), the program reports all 
values \(i\) and \(\Mup{i}\) if \(\Mup{i}\geq\ell\). When using the
option \Showoption{ms}, for each unit \(s\) and each positions \(i\) in
\(s\), the program reports all values \(i\) and \(\MS{i}\) if the length
of \(\MS{i}\) is at lest \(\ell\).
}

\Option{max}{$\ell$}{
Specify the maximum length $\ell$ of minimum unique prefixes or matching
statistics length to be reported.
That is, in the default case,
for each unit \(s\) and each positions \(i\) in \(s\), the program reports all 
values \(i\) and \(\Mup{i}\) if \(\Mup{i}\leq\ell\). When using the
option \Showoption{ms}, for each unit \(s\) and each positions \(i\) in
\(s\), the program reports all values \(i\) and \(\MS{i}\) if the length
of \(\MS{i}\) is at most \(\ell\).
}

\Option{output}{(\Showoptionkey{subjectpos}$\mid$\Showoptionkey{querypos}$\mid$\Showoptionkey{sequence})}{
Specify what to output. At least one of the three keys words
$\Showoptionkey{subjectpos}$,
$\Showoptionkey{querypos}$, and
$\Showoptionkey{sequence}$ must be used. The keyword
$\Showoptionkey{subjectpos}$ is only possible in combination with option
\Showoption{ms}. Using the keyword $\Showoptionkey{subjectpos}$ shows the 
subject position of the matching statistics.
Using the keyword $\Showoptionkey{querypos}$ shows the query position.
Using the keyword $\Showoptionkey{sequence}$ shows the sequence content
of the minimum unique prefix.
}

\Helpoption

\end{Justshowoptions}
The following conditions must be satisfied:
\begin{enumerate}
\item
If both options \Showoption{min} and \Showoption{max} are used, then
the value specified by option \(\Showoption{min}\) must be smaller
than the value specified by option \(\Showoption{max}\).
\end{enumerate}

\section{Examples}

Suppose we have two files \texttt{indexfile.fna} and
\texttt{queryfile.fna}. In the first step, we index \texttt{indexfile.fna}
using the program \MKV:

\EXECUTE{mkvtree -dna -pl -suf -tis -v -db indexfile.fna -indexname indexname}

We obtain one index \texttt{indexname}.
This is used in the following call to the program \GRM:

\EXECUTE{uniquesub -output subjectpos querypos indexname queryfile.fna}

For all units \(s\) in the multiple \Fasta file \texttt{queryfile.fna},
a line is shown, reporting the corresponding value \(\Lmin\).
Also, all positions \(i\) in \(s\) with \(\Mup{i}=\Lmin\)
are reported, together with the corresponding unique position
in the index, shown in the first two columns. The first column is the
sequence number in the index. The second number is the relative position
in this sequence. The \symbol{43}-sign prepended to this position
means that the unique occurrence is on the forward strand of the
indexed sequences. The \(\Lmin\)-value for both units is 3. In the first
unit, there are four positions with a minimum unique prefix of length 3,
namely position 9, 13, 31, and 82.
In the second unit, there is only one position with a minimum unique prefix
of length 3, namely position 12.

To additionally report the sequence content of the
minimum unique prefixes we add option \Showoption{s}.

\EXECUTE{uniquesub -output subjectpos querypos sequence indexname queryfile.fna}

Instead of reporting positions with minimum length prefix over a
unit we can report all \(\Mup{i}\) values in a
given range, using the options \Showoption{min} and \Showoption{max}.
For example, the following program call reports all positions
with a maximal unique prefix of length 5 or longer, again
with the corresponding unique sequence.

\EXECUTE{uniquesub -min 5 -output querypos sequence indexname queryfile.fna}

In each line showing the unique substrings, the first first number is the 
query position and the second is the length. Note that we have omitted
the argument $\Showoptionkey{subjectpos}$ to option \Showoption{output}
to suppress the subject positions.

An upper bound on the length of the minimum unique prefix can be specified 
by the option \Showoption{max}.

\EXECUTE{uniquesub -max 4 -output querypos sequence indexname queryfile.fna}

Of course, the options \Showoption{min} and \Showoption{max}
can be combined. 

Next we create an index \texttt{indexname-bothstrands.rcm}
for both strands of the sequences in \texttt{indexfile.fna}:

\EXECUTE{mkrcidx -v -db indexfile.fna -indexname indexname-bothstrands}

Note that the suffix \texttt{.rcm} is automatically appended to
the given indexname.

We run \GRM on this index and obtain sequences which are
unique on both strands. 

\EXECUTE{uniquesub -output subjectpos querypos sequence indexname-bothstrands.rcm queryfile.fna}

Note that several of the reported sequences are unique on the reverse
strands, which is signified by prepending the symbol \symbol{45} to the
relative subject position.

\section{Change history}

\Showrecentchange{2006-09-22}{\GRM}
The program now also handles reverse complemented indexes to allow
computing strings which are unique both on the forward and the reverse
strand.

\Showrecentchange{2006-09-22}{\GRM}
The option \Showoption{s} has been deleted. Instead use the
option \Showoption{output} with argument $\Showoptionkey{sequence}$.
The option \Showoption{output} also allows to specify that
subject positions and query positions are to be output. In the default
case, neither the sequence, nor the positions are output.

\Showrecentchange{2006-09-18}{\GRM}
If a unit does not contain a unique substring, then the program
now reports this explicitely by a line of the following form:
\begin{verbatim}
unit 13: no unique substrings
\end{verbatim}

\Showrecentchange{2006-09-18}{\GRM}
Besides the unit number, the program also reports the description of the
query sequence, as stored in the fasta header.

\Showrecentchange{2006-09-18}{\GRM}
The program now also accepts a gzipped query file.

\end{document}
