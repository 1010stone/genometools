\documentclass[12pt,titlepage]{article}
\usepackage[a4paper,top=20mm,bottom=20mm,left=20mm,right=20mm]{geometry}
\usepackage{url}
\usepackage{alltt}
\usepackage{xspace}
\usepackage{times}
\usepackage{listings}
\usepackage{bbm}
\usepackage{verbatim}
\usepackage{optionman}

\newcommand{\LTRharvest}{\textit{LTRharvest}\xspace}
\newcommand{\GenomeTools}{\textit{GenomeTools}\xspace}
\newcommand{\Gtltrprog}{\texttt{gt ltrharvest}\xspace}
\newcommand{\Gt}{\texttt{gt}\xspace}
\newcommand{\Gtsuffixerator}{\texttt{gt suffixerator}\xspace}
\newcommand{\Suffixerator}{\texttt{suffixerator}\xspace}

\title{LTRharvest\\
a manual}
\author{\begin{tabular}{c}
         \emph{David Ellinghaus}\\
         Center for Bioinformatics,\\
         University of Hamburg
        \end{tabular}}
\begin{document}
\maketitle

\section{Introduction} \label{Introduction}

This document describes \LTRharvest, a software tool
for \textit{de novo} predictions of LTR 
retrotransposons in genomic sequences~\cite{EKW07}.
\LTRharvest computes boundary positions of potentially 
LTR retrotransposons
on a persistent index structure of the genomic target sequence, 
the \textit{enhanced suffix array} \cite{AKO04}. Usually
the genomic target sequence is a complete chromosomal DNA sequence
in FASTA format. Nevertheless, every DNA sequence in (multiple) 
FASTA format can be passed to the software. For the
prediction, \LTRharvest implements several filters.
These are consecutively applied on the sequence
data to reject candidates, which are not conform with sequence,
length or distance features of LTR retrotransposons. 
Since these features are mostly species-specific,
every filter can be switched on or switched off and is free for 
parameterisation of a certain LTR retrotransposon model.

\LTRharvest is written in \texttt{C} and it is based 
on the \GenomeTools library \cite{genometools}. \LTRharvest is called
as part of the single binary named \Gt.
%The source code is single threaded and can be compiled on 32-bit and 64-bit 
%platforms without making any changes to the sources.
\LTRharvest runs on an enhanced suffix array index, which is stored on files.
This index needs to be constructed by the program \texttt{suffixerator}, which
is also part of the \GenomeTools binary \Gt.

\section{Usage} \label{Usage}

Some text is highlighted by different fonts according to the following rules.

\begin{itemize}
\item \texttt{Typewriter font} is used for the names of software tools.
\item \texttt{\small{Small typewriter font}} is used for file names.
\item \begin{footnotesize}\texttt{Footnote sized typewriter font}
      \end{footnotesize} with a leading 
      \begin{footnotesize}\texttt{'-'}\end{footnotesize} 
      is used for program options.
\item \Showoptionarg{small italic font} is used for the argument(s) of an
      option.
\end{itemize}

\subsection{The options of \LTRharvest} \label{Overview}

Since \LTRharvest is part of \Gt, \LTRharvest is called as follows.

\Gtltrprog \Showoption{index} \Showoptionarg{indexname} $[$\emph{options}$]$

where \Showoptionarg{indexname} denotes the enhanced suffix array index of 
the target sequence(s) constructed by \Gtsuffixerator. 
An overview of all possible options 
with a short one-line description of each option is given in Table
\ref{overviewOpt}.
All options can be specified only once.

\begin{table}[htbp]
\caption{Overview of the \LTRharvest options sorted by categories.}
\begin{footnotesize}
\[
\renewcommand{\arraystretch}{0.89}
\begin{tabular}{ll}\hline
\Showoptiongroup{Input options}
\Showoption{index}& specify the name of the enhanced suffix array index
\\
\Showoptiongroup{Output options}
\Showoption{out}& specify outputfilename for predictions in multiple FASTA
format
\\
\Showoption{gff3}& specify outputfilename for predictions in GFF3 format
\\
\Showoption{outinner}& specify outputfilename for multiple FASTA file 
of the inner regions of predictions 
\\
\Showoptiongroup{Filter options}
\Showoption{seed}& specify minimum seed length for exact maximal repeats
\\
\Showoption{minlenltr}& specify minimum length for each LTR
\\
\Showoption{maxlenltr}& specify maximum length for each LTR
\\
\Showoption{mindistltr}& specify minimum distance of LTR
startpositions
\\
\Showoption{maxdistltr}& specify maximum distance of LTR
startpositions  
\\
\Showoption{similar}& specify similaritythreshold in range [1..100\%]
\\
\Showoption{mintsd}& specify minimum length for each TSD
\\
\Showoption{maxtsd}& specify maximum length for each TSD (requires 
\Showoption{mintsd}) 
\\
\Showoption{motif}& specify palindromic motif consisting of 2 nucleotides 
startmotif + 2 nucleotides endmotif,
\\
\Showoption{motifmis}& specify maximum number of mismatches in motif 
(requires \Showoption{motif}) 
\\
\Showoption{vic}& specifiy the number of nucleotides (to the left and to 
the right)
\\
& that will be searched for TSDs and/or motifs around 5' and 3' boundary
\\
\Showoption{overlaps}& specify no$|$best$|$all
\\
\Showoptiongroup{Alignment options}
\Showoption{xdrop}& specify xdrop for extension-alignment
\\
\Showoption{mat}& specify matchscore for extension-alignment
\\
\Showoption{mis}& specify mismatchscore for extension-alignment
\\
\Showoption{ins}& specify insertionscore for extension-alignment
\\
\Showoption{del}& specify deletionscore for extension-alignment
\\
\Showoptiongroup{Miscellaneous options}
\Showoption{v}& verbose mode
\\
\Showoption{longoutput}& additional motif/TSD output (requires 
\Showoption{mintsd} or \Showoption{motif})
\\
\Showoption{help}& show all options
\\
\hline
\end{tabular}
%\input{ltrprogoptions}
\]
\end{footnotesize}
\label{overviewOpt}
\end{table}

%%%%
\subsection{Input options}

\begin{Justshowoptions}

\Option{index}{\Showoptionarg{indexname}}{
Specify the name of the enhanced suffix array index. The index must comprise
the tables $\tt{suf}$, $\tt{lcp}$, $\tt{des}$ and
$\tt{tis}$. An example construction of the enhanced suffix array index
with \Gtsuffixerator is shown is section~\ref{Examples}. 
%The instruction of
%\Gtsuffixerator is included in the \emph{Vmatch manual}~\cite{vmatch}.
}
\end{Justshowoptions}

The option \Showoption{index} is mandatory.

%%%%
\subsection{Output options}

Results are reported in tabular fashion on stdout and can easily
be written to a file notation \texttt{\symbol{62} resultfile} as in
the following:

\Gtltrprog \Showoption{index} \Showoptionarg{indexname} 
$[$\emph{options}$]$ \texttt{\symbol{62}} \Showoptionarg{resultfile} 

In addition the following options write extra information to files
\begin{Justshowoptions}

\Option{out}{\Showoptionarg{outputfile}}{
Specify the name of a file where the predictions will be written to.
Each prediction will be represented by an individual FASTA entry.
}

\Option{outinner}{\Showoptionarg{outputfile}}{
Specify the name of a file where the inner regions - the prediction sequences 
without the flanking LTR sequences - will be written to.
Each prediction will be represented by an individual FASTA entry. 
}

\Option{gff3}{\Showoptionarg{outputfile}}{
Specify the name of a file where the predictions will be written to.
Each prediction will be represented by an individual GFF3~\cite{gff3} entry.
}

\end{Justshowoptions}


%%%%
\subsection{Filter options}

The filter options provide the opportunity to exclude predictions with unwanted
sequence, length or distance features. If a particular option is not set 
by the user, a default value for this options will be set, 
except for the options \Showoption{mintsd}, \Showoption{maxtsd} 
and \Showoption{motif}. Thus, if none of the filter options is set by the user,
a prediction of LTR retrotransposons will be
conducted without searching for target site duplications (TSD) or a particular
LTR start-end motif.

\begin{Justshowoptions}

\Option{seed}{\Showoptionarg{$L_{ex}$}}{
Specify the minimum length for the exact maximal repeats. Only those repeats
with the specified minimum length are analyzed in the process of finding 
candidate pairs. Exact maximal repeats of length below that threshold are 
not taken further into account. 
\Showoptionarg{$L_{ex}$} has to be a positive integer. If this
option is not selected by the user, then \Showoptionarg{$L_{ex}$} is set
to $30$ by default.
}

\Option{minlenltr}{\Showoptionarg{$L_{min}$}}{ 
Specify the minimum length of each LTR. \Showoptionarg{$L_{min}$}
has to be specified as positive integer.
If this option is not selected by the user, then \Showoptionarg{$L_{min}$} 
is set to $100$ by default.
}

\Option{maxlenltr}{\Showoptionarg{$L_{max}$}}{ 
Specify the maximum length of each LTR. \Showoptionarg{$L_{max}$}
has to be specified as positive integer.
If this option is not selected by the user, then \Showoptionarg{$L_{max}$} 
is set to $1000$ by default.
}

\Option{mindistltr}{\Showoptionarg{$D_{min}$}}{ 
Specify the minimum distance of LTR starting positions.
\Showoptionarg{$D_{min}$} has to be specified as positive integer.
If this option is not selected by the user, then \Showoptionarg{$D_{min}$} 
is set to $1000$ by default.
}

\Option{maxdistltr}{\Showoptionarg{$D_{max}$}}{ 
Specify the maximum distance of LTR starting positions.
\Showoptionarg{$D_{max}$} has to be specified as positive integer.
If this option is not selected by the user, then \Showoptionarg{$D_{max}$} 
is set to $15000$ by default.
}

\Option{similar}{\Showoptionarg{similaritythreshold}}{ 
Specify the minimum similarity value between the two LTRs.
The argument \Showoptionarg{similaritythreshold} has to be chosen 
from the range [0,100] and means a percentage.
If this option is not selected by the user, the default value is 
set to 85\%.
}

\Option{mintsd}{\Showoptionarg{TSDminlen}}{
If this option is selected, a search for target site duplications 
(TSDs) will be conducted with a minimum TSD length of 
\Showoptionarg{TSDminlen}. 
If this option is not selected by the user, a search for TSDs
will not be conducted.
If this option is set but no maximum TSD length is specififed by the
option \Showoption{maxtsd}, then the maximum TSD length is set to $20$
by default.
}

\Option{maxtsd}{\Showoptionarg{TSDmaxlen}}{
This option requires the option \Showoption{mintsd}.
If this option is selected, a search for target site duplications 
(TSDs) will be conducted with a maximum TSD length of 
\Showoptionarg{TSDmaxlen}. 
}

\Option{motif}{\Showoptionarg{expr}}{
Specify 2 nucleotides for the starting motif and 2 nucleotides for 
the ending motif at the beginning and the ending of each LTR,
respectively.
Only palindromic motif sequences - where the motif sequence
is equal to its complementary sequence read backwards -
%- of
%nucleotides a(adenine), c(cytosine), g(guanine), t(thymine) 
are allowed, e.g. \Showoptionarg{tgca}.
Type the nucleotides without any space separating them.
If this option is not selected by the user, then candidate pairs will not
be checked, if they contain a motif.
If this options is set but no allowed number of mismatches is 
specified by the option \Showoption{motifmis}, then a search for
the exact motif will be conducted.
}

\Option{motifmis}{\Showoptionarg{n}}{
This option requires the option \Showoption{motif}.
Specify the allowed number of mismatches by the argument 
\Showoptionarg{n}. If this option is not set, then a search for
the exact motif will be conducted. The non-negative integer
\Showoptionarg{n} has to be chosen from range $[0,3]$.
}

\Option{vic}{\Showoptionarg{l}}{
Specify the number of (vicinity) nucleotides \Showoptionarg{l}
(to the left and to the right, respectively) that
will be searched for TSDs and/or one motif around the 
5' and 3' predicted boundary of a LTR retrotransposon. This option
has only an effect, if option \Showoption{mintsd} and/or option
\Showoption{motif} is switched on.
If this option is not selected by the user, the default value of $l$
is 60.
}

\Option{overlaps}
{\Showoptionarg{no}$|$\Showoptionarg{best}$|$\Showoptionarg{all}}{
Specify the output with regard to nested and/or overlapping 
LTR retrotransposon predictions. If the argument 
\Showoptionarg{no} is selected, then neither nested nor overlapping 
predictions will be reported in the output. 
If the argument \Showoptionarg{best} is
selected, then, in the case of two or more
nested or overlapping predictions, 
solely the LTR retrotransposon prediction with the highest 
similarity between its LTRs will be reported.
If the argument \Showoptionarg{all} is
selected, then all LTR retrotransposon predictions will be reported
whether there are nested and/or overlapping predictions or not.
If this option is not selected by the user,
the option with argument \Showoptionarg{best} is set by default. 
}

\end{Justshowoptions}


%%%%
\subsection{Alignment options}

An $X$-drop extension process permits the search for degenerated LTRs.
The alignment options provide the opportunity to control this extension 
of the maximal repeat seeds. 
If a particular alignment option is not set 
by the user, a default value for this options will be set.

\begin{Justshowoptions}

\Option{xdrop}{\Showoptionarg{X}}{ 
Specify the xdrop \Showoptionarg{X} for extending a seed repeat in both
directions allowing for matches, mismatches, insertions and deletions. The
argument \Showoptionarg{X} must be a positive integer or $0$. 
The $X$-drop extension process stops as soon as the extension involving matches,
mismatches, insertions, and deletions has a score smaller than $T-X$ where T
denotes the largest score seen so far.
If this option
is not selected by the user, then \Showoptionarg{X} is set to $5$ by default.
}

\Option{mat}{\Showoptionarg{score}}{
Specify the positive match score for the $X$-drop extension process. If the
option is not selected by the user, the default value is $2$.
}

\Option{mis}{\Showoptionarg{score}}{
Specify the negative mismatch score for the $X$-drop extension process. 
If this option is not selected by the user, the default value is $-2$.
}

\Option{ins}{\Showoptionarg{score}}{
Specify the negative insertion score for the $X$-drop extension process. 
If this option is not selected by the user, the default value is $-3$.
}

\Option{del}{\Showoptionarg{score}}{
Specify the negative deletion score for the $X$-drop extension process. 
If this option is not selected by the user, the default value is $-3$.
}

\end{Justshowoptions}

%%%%
\subsection{Miscellaneous options}

\begin{Justshowoptions}

\Option{v}{}{
This option enables the verbose mode. This means, that some more information
about the processing will be printed to \texttt{stdout} during the run.
This includes a long list of switched on or switched off options.
}

\Option{longoutput}{}{
This option additionally prints information about the detected 
TSD and/or the motif to \texttt{stdout}, 
if a search for TSD and/or for the motif
has been selected by the user.
This option requires the option \Showoption{mintsd} and/or
\Showoption{motif}.
}

\Option{help}{}{
\LTRharvest will show a summary of all options on
\texttt{stdout} and terminate with exit code $0$.
}

\end{Justshowoptions}
%%%%%%%%%%%%

\section{Examples} \label{Examples}

In this section, example applications of \LTRharvest are presented. 
In Subsection~\ref{different-options}, examples for using different options of 
\LTRharvest are given. Subsection~\ref{Example-genome} then gives an example 
for the prediction of LTR
retrotransposons on the entire \textit{S. cerevisiae} genome.
In Subsection~\ref{ExampleCluster}, an example for a clustering process of 
the \LTRharvest output is shown. Please note that this step is not part of 
\LTRharvest and is carried out by
%with a subsequent
%clustering process of the predicted sequences using 
\textit{Vmatch}~\cite{vmatch}.
%(which is not part of the \GenomeTools binary \Gt). 

\subsection{Using different options of \LTRharvest}
\label{different-options}

As target
sequence file we use some FASTA file \texttt{\small{chr02.19970727.fsa.gz}} 
containing the \textit{S. cerevisiae} genome sequence, chromosome 2. Note
that the file is a compressed file in gzip format (because of the suffix 
\texttt{.gz}. This format can easily be handled by the program \Suffixerator.

First of all, we create the enhanced suffix array index using \Suffixerator.
We invoke \Suffixerator with options \Showoption{tis}, \Showoption{suf}, 
\Showoption{lcp}, \Showoption{des}, since \LTRharvest needs the corresponding
tables. Furthermore, we specify \Showoption{dna} for using a DNA-alphabet.

\begin{footnotesize}
\begin{verbatim}
$ gt suffixerator -db chr02.19970727.fsa.gz -indexname chr02.19970727.fsa -tis 
-suf -lcp -des -dna
# dna=yes
# indexname="chr02.19970727.fsa"
# prefixlength=automatic
.
.
.
# TIME overall 1.37
\end{verbatim}
\end{footnotesize}

Now we can use the index with \LTRharvest. The first example call will just use
the default parameters for the filter and alignment options 
without searching for TSD or a LTR start-end motif.

\begin{footnotesize}
\begin{verbatim}
$ gt ltrharvest -index chr02.19970727.fsa
# args=-index chr02.19970727.fsa
# predictions are reported in the following way
# s(ret) e(ret) l(ret) s(lLTR) e(lLTR) l(lLTR) s(rLTR) e(rLTR) l(rLTR) 
sim(LTRs) seq-nr
# where:
# s = starting position
# e = ending position
# l = length
# ret = LTR-retrotransposon
# lLTR = left LTR
# rLTR = right LTR
# sim = similarity
# seq-nr = sequence number
259532  265448  5917  259532  259863  332  265117  265448  332  99.40  0
427672  430021  2350  427672  428170  499  429522  430021  500  91.60  0
29632  35597  5966  29632  29969  338  35259  35597  339  98.82  0
220989  226919  5931  220989  221339  351  226574  226919  346  97.15  0
\end{verbatim}
\end{footnotesize}

Each comment line starts with the comment symbol \#.
Each non-comment line denotes a LTR retrotransposon prediction with
starting and ending positions of the whole LTR retrotransposon, the
left LTR instance and the right LTR instance, respectively. Furthermore,
for each of these elements, the corresponding element length is reported
as well as a percentage similarity of the two LTRs. The last integer of
each line denotes the number of the input sequence, the LTR retrotransposon
prediction occurs in. The input sequence numbers are counted from 0.

Invoking \LTRharvest with the optional argument \Showoption{v} gives
more information about enabled and disabled options as well as 
additional information about the enhanced suffix array index
and time/space consumption. Moreover, specifying options \Showoption{out} 
and \Showoption{outinner} the run results in two multiple FASTA files.

\begin{footnotesize}
\begin{verbatim}
$ gt ltrharvest -index chr02.19970727.fsa -v -out pred-chr02.fsa 
-outinner pred-inner-chr02.fsa
# args=-index chr02.19970727.fsa -v -out pred-chr02.fsa -outinner 
pred-inner-chr02.fsa
# user defined options and values:
#   verbosemode: On
#   indexname: chr02.19970727.fsa
#   outputfile: pred-chr02.fsa
#   outputfile inner region: pred-inner-chr02.fsa
#   xdropbelowscore: 5
#   similaritythreshold: 85.00
#   minseedlength: 30
#   matchscore: 2
#   mismatchscore: -2
#   insertionscore: -3
#   deletionscore: -3
#   minLTRlength: 100
#   maxLTRlength: 1000
#   minLTRdistance: 1000
#   maxLTRdistance: 15000
#   overlaps: best
#   minTSDlength: 0
#   maxTSDlength: 20
#   palindromic motif:
#   motifmismatchesallowed: 4
#   vicinity: 60 nt
# predictions are reported in the following way
# s(ret) e(ret) l(ret) s(lLTR) e(lLTR) l(lLTR) s(rLTR) e(rLTR) l(rLTR) sim(LTRs) 
seq-nr
# where:
# s = starting position
# e = ending position
# l = length
# ret = LTR-retrotransposon
# lLTR = left LTR
# rLTR = right LTR
# sim = similarity
# seq-nr = sequence number
259532  265448  5917  259532  259863  332  265117  265448  332  99.40  0
427672  430021  2350  427672  428170  499  429522  430021  500  91.60  0
29632  35597  5966  29632  29969  338  35259  35597  339  98.82  0
220989  226919  5931  220989  221339  351  226574  226919  346  97.15  0
\end{verbatim}
\end{footnotesize}

Searching additionally for TSD and a LTR start-end motif we use
the options \Showoption{mintsd}, \Showoption{maxtsd}, \Showoption{motif} and 
\Showoption{motifmis}.

\begin{footnotesize}
\begin{verbatim}
$ gt ltrharvest -index chr02.19970727.fsa -mintsd 5 -maxtsd 20 -motif tgca
-motifmis 0
# args=-index chr02.19970727.fsa -mintsd 5 -maxtsd 20 -motif tgca -motifmis 0
# predictions are reported in the following way
# s(ret) e(ret) l(ret) s(lLTR) e(lLTR) l(lLTR) s(rLTR) e(rLTR) l(rLTR) sim(LTRs)
seq-nr
# where:
# s = starting position
# e = ending position
# l = length
# ret = LTR-retrotransposon
# lLTR = left LTR
# rLTR = right LTR
# sim = similarity
# seq-nr = sequence number
259532  265448  5917  259532  259863  332  265117  265448  332  99.40  0
29632  35590  5959  29632  29963  332  35259  35590  332  99.70  0
220996  226911  5916  220996  221329  334  226575  226911  337  97.33  0
\end{verbatim}
\end{footnotesize}

Finally, if we are interested in the sequence and the length of the TSD
as well as the sequence of the motif, we select the option 
\Showoption{longoutput}.
This also is an example for specifying all filter and alignment 
options by the user.

\begin{footnotesize}
\begin{verbatim}
$ gt ltrharvest -index chr02.19970727.fsa -seed 30 -xdrop 5 -mat 2 -mis -2 -ins -3
-del -3 -minlenltr 100 -maxlenltr 1000 -mindistltr 1000 -maxdistltr 15000
-similar 90.0 -overlaps all -mintsd 5 -maxtsd 20 -motif tgca -motifmis 0 -vic 60
-longoutput
# args=-index chr02.19970727.fsa -seed 30 -xdrop 5 -mat 2 -mis -2 -ins -3 -del -3
-minlenltr 100 -maxlenltr 1000 -mindistltr 1000 -maxdistltr 15000 -similar 90.0
-overlaps all -mintsd 5 -maxtsd 20 -motif tgca -motifmis 0 -vic 60 -longoutput
# predictions are reported in the following way 
# s(ret) e(ret) l(ret) s(lLTR) e(lLTR) l(lLTR) TSD l(TSD) m(lLTR) s(rLTR) 
e(rLTR) l(rLTR) TSD l(TSD) m(rLTR) sim(LTRs) seq-nr
# where:
# s = starting position
# e = ending position
# l = length
# m = motif
# ret = LTR-retrotransposon 
# lLTR = left LTR
# rLTR = right LTR
# TSD = target site duplication
# sim = similarity
# seq-nr = sequence number
259532  265448  5917  259532  259863  332  gtaat  5  tg..ca  265117  265448  332
gtaat  5  tg..ca  99.40  0
29632  35590  5959  29632  29963  332  ataat  5  tg..ca  35259  35590  332
ataat  5  tg..ca  99.70  0
220996  226911  5916  220996  221329  334  ggaat  5  tg..ca  226575  226911 337  
ggaat  5  tg..ca  97.33  0
\end{verbatim}
\end{footnotesize}

\subsection{Predictions on the entire \textit{S.cerevisiae} genome}
\label{Example-genome}

As target sequences file we use some multiple FASTA file
\texttt{\small{chrAll.19971001.fsa}}, which contains all 16
chromosomal sequences of the release before Oct. 1st, 1997, probably 
used by Kim et al.
in their comprehensive survey of retrotransposons~\cite{kim:1998}.

First of all, we create the enhanced suffix array index using 
\Suffixerator.
We invoke \Suffixerator with options \Showoption{tis}, \Showoption{suf}, 
\Showoption{lcp}, \Showoption{des}, since \LTRharvest needs the corresponding
tables. Furthermore, we specify \Showoption{dna} for using a DNA-alphabet.

\begin{footnotesize}
\begin{verbatim}
$ gt suffixerator -db chrAll.19971001.fsa.gz -indexname chrAll.19971001.fsa -tis 
-suf -lcp -des -dna
# dna=yes
# indexname="chrAll.19971001.fsa"
# prefixlength=automatic
.
.
.
# TIME overall 108.35
\end{verbatim}
\end{footnotesize}

Now we can use the index with \LTRharvest. In addition to the filter and 
alignment options, we choose option \Showoption{out} for printing the predicted 
LTR retrotransposon sequences to a file.

\begin{footnotesize}
\begin{verbatim}
$ gt ltrharvest -index chrAll.19971001.fsa -seed 100 -minlenltr 100 -maxlenltr 1000 
-mindistltr 1000 -maxdistltr 15000 -xdrop 5 -mat 2 -mis -2 -ins -3 -del -3 
-similar 90.0 -overlaps best -mintsd 5 -maxtsd 20 -motif tgca -motifmis 0 
-vic 60 -longoutput 
-out pred-chrAll.fsa
# args=-index chrAll.19971001.fsa -seed 100 -minlenltr 100 -maxlenltr 1000 
-mindistltr 1000 -maxdistltr 15000 -xdrop 5 -mat 2 -mis -2 -ins -3 -del -3 
-similar 90.0 -overlaps best -mintsd 5 -maxtsd 20 -motif tgca -motifmis 0 
-vic 60 -longoutput -out pred-chrAll.fsa                                                
# predictions are reported in the following way 
# s(ret) e(ret) l(ret) s(lLTR) e(lLTR) l(lLTR) TSD l(TSD) m(lLTR) s(rLTR) e(rLTR) 
l(rLTR) TSD l(TSD) m(rLTR) sim(LTRs) seq-nr                               
# where:                                                                                                                                                    # s = starting position                                                                                                                                     # e = ending position                                                                                                                                       # l = length                                                                                                                                                # m = motif                                                                                                                                                 # ret = LTR-retrotransposon                                                                                                                                 # lLTR = left LTR                                                                                                                                           # rLTR = right LTR                                                                                                                                          # TSD = target site duplication                                                                                                                             # sim = similarity                                                                                                                                          # seq-nr = sequence number
160239  166163  5925  160239  160575  337  ggttc  5  tg..ca  165827  166163  337  
ggttc  5  tg..ca  100.00  0
29632  35590  5959  29632  29963  332  ataat  5  tg..ca  35259  35590  332  
ataat  5  tg..ca  99.70  1
.
.
.
844407  850335  5929  844407  844744  338  gaaat  5  tg..ca  849998  850335  338  
gaaat  5  tg..ca  100.00  15
56452  62375  5924  56452  56788  337  gttat  5  tg..ca  62039  62375  337  
gttat  5  tg..ca  100.00  15
\end{verbatim}
\end{footnotesize}

\subsection{Sequence clustering of \LTRharvest output (optional)}
\label{ExampleCluster}

In addition to the prediction process done by \LTRharvest, 
a cluster analysis on the predicted
sequences is recommended. Here, we choose the single linkage cluster 
analysis program from the
\textit{Vmatch} software tool~\cite{vmatch} 
(which is not part of the \GenomeTools binary \Gt)
in order to show how this task can be accomplished. 
An index needs to be constructed from the predicted sequences by the program
\texttt{mkvtree} which is part of \textit{Vmatch}.

\begin{footnotesize}
\begin{verbatim}
$ mkvtree -db pred-chrAll.fsa -dna -pl -allout -v
reading file "pred-chrAll.fsa"
for this input size and alphabet size, the maximal prefixlength
  (argument of option -pl) is 8,
  the recommended prefixlength is 7
  total length of sequences: 273557 (including 45 separators)
  alphabet "aAcCgGtTuUnsywrkvbdhmNSYWRKVBDHM" (size 32) mapped to "acgtn" (size 5)
  create file "pred-chrAll.fsa.tis"
.
.
.
overall space peak: main=2.61 MB (10.01 bytes/symbol), secondary=0.53 MB
\end{verbatim}
\end{footnotesize}

Now we can use \texttt{vmatch} (which is part of \textit{Vmatch}) 
to perform a clustering of the predicted sequences.
A default parameter set can be the one
from Tab.~\ref{parameters-cluster-default}, 
page~\pageref{parameters-cluster-default}. 
The instruction of \texttt{vmatch}
is included in the \emph{Vmatch manual}~\cite{vmatch}.
The following command computes clusters of all (database) sequences in 
\texttt{\small{pred-chrAll.fsa}}. Each cluster has a unique cluster number
with the (database) sequence numbers listed behind. 

\begin{footnotesize}
\begin{verbatim}
$ vmatch -dbcluster 95 7 Cluster-pred-chrAll -p -d -seedlength 50
-l 1101 -exdrop 9 pred-chrAll.fsa
# args=-dbcluster 95 7 Cluster-pred-chrAll -p -d -seedlength 50
-l 1101 -exdrop 9 pred-chrAll.fsa
# 3 clusters
# 45 elements out of 46 (97.83%) are in clusters
# 1 elements out of 46 (2.17%) are singlets
# 1 cluster of size 2
# 1 cluster of size 3
# 1 cluster of size 40
0:  32 41 42 36 39 13 40 43 37 30 44 38 18 21 17 34 35 31 12 20
33 23 26 27 6 28 3 29 25 1 24 9 19 2 7 22 4 10 5 0
1:  8 15 11
2:  14 45
\end{verbatim}
\end{footnotesize}

\bibliographystyle{plain}
\bibliography{references}

\begin{table}
\caption{A default parameter set for \textit{Vmatch}'s single linkage 
  clustering program.}
\vspace{0.25cm}
\begin{tabular}[h!]{lcl}\hline
Parameter name & Value & Comment\\\hline
dbcluster    & 95 ($\lfloor(\frac{D_{min}}{D_{max}} \times 100)\rfloor + 1$) & Match covers at least 95\% of the smaller sequence and\\
             &                                                 & (\textit{formula value})\% of the larger sequence\\
d  & & Compute direct matches\\
p & & Compute palindromic matches\\
seedlength & 50 & Minimal length of the exact repeats\\
l & $D_{min} + L_{min}$ & Minimal length of matches\\
exdrop & 9 & Xdrop score when extending a seed in both directions\\\hline
\end{tabular}
\label{parameters-cluster-default}
\end{table}
\end{document}
