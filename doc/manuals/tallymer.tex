\documentclass[12pt]{article}
\usepackage[a4paper,top=20mm,bottom=20mm,left=20mm,right=20mm]{geometry}
\usepackage{url}
\usepackage{alltt}
\usepackage{xspace}
\usepackage{times}
\usepackage{listings}
\usepackage{bbm}
\usepackage{verbatim}
%\usepackage{prognames}
\usepackage{optionman}
\newcommand{\Filenamesuffix}[1]{\texttt{\small #1}\index{#1@\texttt{#1}}}
\newcommand{\Programname}[1]{\texttt{\small #1}}
\newcommand{\Tallymer}[0]{\Programname{tallymer}\xspace}
\newcommand{\TYmkindex}[0]{\Programname{mkindex}\xspace}
\newcommand{\TYsearch}[0]{\Programname{search}\xspace}
\newcommand{\TYoccratio}[0]{\Programname{occratio}\xspace}
\newcommand{\SFX}[0]{\Programname{suffixerator}\xspace}
\newcommand{\SFXidx}[0]{\textit{suffixerator}-index\xspace}
\newcommand{\Tyridx}[0]{\textit{tallymer}-index\xspace}
\newcommand{\GT}[0]{\Programname{gt}\xspace}
\newcommand{\Kmin}[0]{k_{\min}}
\newcommand{\Kmax}[0]{k_{\max}}
\newcommand{\Kstep}[0]{k_{step}}
\newcommand{\Minmaxocctext}[2]{
Specify the #1\xspace occurrence number for which to output the $k$-mer
sequences. That is, a $k$-mer is output, if it
occurs #2\xspace $c$ times in the union of all sequences from the
\SFXidx. When combined with option \Showoption{indexname},
this option specifies an occurrence constraint on the $k$-mers stored
in the generated \Tyridx. That is, a $k$-mer is put into the
\Tyridx, if it occurs #2\xspace $c$ times in the union of
all sequences from the \SFXidx.
}
\newcommand{\Typrogintro}[1]{
\par
\noindent\GT \Programname{tallymer} #1\xspace [\textit{options}] \Showoption{esa}
\SFXidx [\textit{options}]
\par
The \SFXidx is an enhanced suffix array computed by the
program \SFX, which is also part of the genometools-package. Currently,
there is no \SFX-manual available, but in Section~\ref{Examples} we show how
to call \SFX appropriately in  the context of
\Tallymer. The following options are available in #1:}

\newcommand{\Esaoption}[0]{
\Option{esa}{\Showoptionarg{\SFXidx}}{
Specify the name of an \SFXidx computed by the
\SFX-program using the output options \Showoption{suf}, \Showoption{lcp},
and \Showoption{tis}. This option is mandatory.
}}

\newcommand{\Scanoption}[0]{
\Option{scan}{}{
Sequentially read the \SFXidx.
In the default case, the \SFXidx is mapped into main memory. This means
that \SFXidx must not be larger than the available
address space. So for a 32-bit machine, the index cannot be larger
than 4~GB. When using this option, the \textsf{lcp}-table and the
\textsf{suf}-table of the \SFXidx are
sequentially scanned, so that only a small part of these tables reside
in memory. This, of course, reduces the memory requirement of
the program. Note that the sequence is still mapped completely into main
memory as it is accessed in random order.}
}

\newcommand{\Standardoptions}[0]{
\Option{v}{~~~}{%
Be verbose, that is, give reports about the different steps as well as the
resource requirements of the computation.
}

\Option{version}{}{
Show the version of the program and exit.
}

\Option{help}{}{
display help and exit.
}
}

\title{The tallymer software for counting, indexing, and searching $k$-mers\\
a manual}
\author{Stefan Kurtz}
\author{\begin{tabular}{c}
         \textit{Stefan Kurtz}\\
         Center for Bioinformatics,\\
         University of Hamburg
        \end{tabular}}

\begin{document}
\maketitle
This manual describes the \textit{Tallymer}-software, a collection of programs
for counting, indexing, and searching $k$-mers. For an introduction of
the notions, concepts, and methods underlying the software, we refer
the reader to \cite{KUR:NER:STE:WAR:2008}. \textit{Tallymer} is part of
the genometools software (\url{http://genometools.org}). It is implemented
as part of the \textit{gt}-binary and called as a subprogram. So to
run \Tallymer, one has to call the \textit{gt}-binary with subprogram
\texttt{tallymer}. \Tallymer itself has three subprograms
\TYmkindex, \TYoccratio, and \TYsearch. These are described below.

\section{\TYmkindex}
The program \TYmkindex is used for counting and indexing \(k\)-mers for
a fixed value of \(k\). It is called as follows:
\par
\Typrogintro{\TYmkindex}

\begin{Justshowoptions}
\Esaoption

\Option{mersize}{$k$}{
Specify the size $k$ of the mers. That is, the program generates all
substrings of length $k$ of the given input sequences, given as a
\SFXidx If this option is missing, then the default value for $k$ is 20.
}

\Option{minocc}{$c$}{
\Minmaxocctext{minimum}{at least}
}

\Option{maxocc}{$c$}{
\Minmaxocctext{maximum}{at most}
}

\Option{pl}{$\Showoptionalarg{prefixlength}$}{
Specify the prefix length to construct a bucket boundary table
for the generated \Tyridx. This additional table speeds up the
search in the \Tyridx. This option only works for an alphabet
of size 4, i.e.\ for the DNA alphabet.
The argument \Showoptionarg{prefixlength} is
optional. Hence it is denoted in square brackets.
If the argument is omitted, then the value
for \Showoptionarg{prefixlength} is automatically determined. More precisely,
it is \(\left\lfloor\log_{4}\frac{n}{4}\right\rfloor\),
where \(n\) is the total number of $k$-mers in the \Tyridx.
}

\Option{indexname}{\Showoptionarg{idxname}}{
Store the  $k$-mers specified according to the options \Showoption{minocc}
and \Showoption{maxocc} in the file named
\Showoptionarg{idxname}\Filenamesuffix{.mer}. If option
\Showoption{pl} is used, then additionally the bucket boundary table is
stored in a file named \Showoptionarg{idxname}\Filenamesuffix{.mbd}.
Using the option \Showoption{counts} (see below), an additional file
\Showoptionarg{idxname}\Filenamesuffix{.mct} is generated. These file
together make up the \Tyridx.
}

\Option{counts}{}{
Specify that
\Showoptionarg{idxname}\Filenamesuffix{.mct}
is generated storing the
counts of the $k$-mers represented by the \Tyridx. This option
can only be used together with option \Showoption{indexname} which also
specifies prefix of the produced output file.
This option is required if the program \TYsearch needs to
report the $k$-mer-counts.
}

\Scanoption

\Standardoptions

\end{Justshowoptions}
The following conditions must be satisfied:
\begin{enumerate}
\item
Option \Showoption{pl} requires to also use option \Showoption{indexname}.
\item
Option \Showoption{counts} requires to also use option \Showoption{indexname}.
\item
Option \Showoption{indexname} requires to also use one of the options
options \Showoption{minocc} and \Showoption{maxocc}.
\end{enumerate}
Note that the program ignores all $k$-mers not entirely consisting of
wildcard characters (i.e. not \texttt{a}, \texttt{c}, \texttt{c}, and
\texttt{g} in case of the DNA alphabet).

\section{\TYoccratio}

The program \TYoccratio is used to compute the occurrence ratios for a set
of sequences represented by a \SFXidx It is called as follows:
\par
\Typrogintro{\TYoccratio}

\begin{Justshowoptions}
\Esaoption

\Option{minmersize}{$\Kmin$}{
Specify the minimum size of the mers which are counted.
That is, the program counts the number of unique and nonunique
mers of length at least $\Kmin$. This option is mandatory if option
\Showoption{mersizes} is not used.}

\Option{maxmersize}{$\Kmax$}{
Specify the maximum size of the mers which are counted.
That is, the program counts the number of unique and nonunique
mers of length at most $\Kmax$. This option is mandatory if option
\Showoption{mersizes} is not used.}

\Option{step}{$\Kstep$}{
Specify the step size according to which the mer counts are output.
That is, for all $k\in[\Kmin,\Kmin+\Kstep,\Kmin+2\Kstep,\ldots,\Kmax]$
the $k$-mer counts are output. If this option is not used, then
$\Kstep$ is 1.}

\Option{mersizes}{$k_{1}~k_{2}\ldots k_{q}$}{
Specify mer sizes $1\leq k_{1}<k_{2}<\cdots<k_{q}$ with $q\geq 1$.}

\Option{output}{(\Showoptionkey{unique}$\mid$\Showoptionkey{nonunique}$\mid$\Showoptionkey{nonuniquemulti}$\mid$\Showoptionkey{relative}$\mid$\Showoptionkey{total})}{
Specify what to output by giving at least one of the four keywords
\begin{center}
$\Showoptionkey{unique}$,
$\Showoptionkey{nonunique}$,
$\Showoptionkey{nonuniquemulti}$,
$\Showoptionkey{relative}$, and
$\Showoptionkey{total}$.
\end{center}
The semantics of the used keywords is a follows:
\begin{description}
\item[$\Showoptionkey{unique}$:]
Show the number of
unique $k$-mers for each $k$ between $\Kmin$ and $\Kmax$.
\item[$\Showoptionkey{nonunique}$:]
Show the number
of non-unique $k$-mers for each $k$ between $\Kmin$ and $\Kmax$.
Only the event that a $k$-mer is unique is counted.
\item[$\Showoptionkey{nonuniquemulti}$:]
Show the
number of non-unique $k$-mers for each $k$ between $\Kmin$ and $\Kmax$.
Each $k$-mer is counted as the number of times it occurs in the indexed
sequences.
\item[$\Showoptionkey{total}$:]
Show the number of all
$k$-mers for each $k$ between $\Kmin$ and $\Kmax$. The distribution
is shown twice, once counting each non-unique $k$-mers as one event,
and once counting each non-unique $k$-mer as the number of times it occurs in
the indexed sequences.
\item[$\Showoptionkey{relative}$:]
Show the fraction of unique/non-unique $k$-mers relative
to all $k$-mers. This keyword can be combined with the keywords
$\Showoptionkey{unique}$, $\Showoptionkey{nonunique}$, and
$\Showoptionkey{nonuniquemulti}$.
\end{description}
}

\Scanoption

\Standardoptions

\end{Justshowoptions}
The following conditions must be satisfied:
\begin{enumerate}
\item
Any of the options \Showoption{minmersize}, \Showoption{maxmersize},
\Showoption{step} cannot be used together with option \Showoption{mersizes}.
\end{enumerate}

\section{\TYsearch}
The program \TYsearch is used to search a set of \(k\)-mers in a
\Tyridx. \TYsearch is called as follows:
\par
\noindent\GT \Programname{tallymer} \TYsearch [\textit{options}] \texttt{-tyr}
\Tyridx~\texttt{-q} \textit{queryfile0} \textit{queryfile1} \dots
\par
where \Tyridx is an index generated by \TYmkindex, and
\textit{queryfile0}, \textit{queryfile1}, etc.\ are queryfiles
(in \Fasta format)
which are to be matches against the given index. The following options are
available:

\begin{Justshowoptions}
\Option{tyr}{\Showoptionarg{\Tyridx}}{
Specify the name of a \Tyridx computed by the
program \TYmkindex. This option is mandatory.
}

\Option{q}{$\Showoptionarg{files}$}{
Specify a white space separated list of query files (in multiple \Fasta
format). At least one query file must be given. The files may be in
gzipped format, in which case they have to end with the suffix \texttt{.gz}.
}

\Option{strand}{(\Showoptionkey{f}$\mid$\Showoptionkey{p}$\mid$\Showoptionkey{fp})}{
Specify the strand to be searched. The keyword \Showoptionkey{f} means to
search on the forward strand, i.e.\ each mer is searched in forward
direction. The keyword \Showoptionkey{p} means to
search on the reverse complemented strand, i.e.\ the reverse complement
of the given mer is searched. The keyword \Showoptionkey{fp} means a
combination of \Showoptionkey{f} and \Showoptionkey{p}.
}

\Option{output}{(\Showoptionkey{qseqnum}$\mid$\Showoptionkey{qpos}$\mid$\Showoptionkey{counts}$\mid$\Showoptionkey{sequence}}{
Specify what to output by giving at least one of the four keywords
\begin{center}
$\Showoptionkey{qseqnum}$,
$\Showoptionkey{qpos}$,
$\Showoptionkey{counts}$,
$\Showoptionkey{sequence}$.
\end{center}
\begin{description}
\item[$\Showoptionkey{qseqnum}$:]
show the sequence number
of the query sequence, the matching mer comes from.
\item[$\Showoptionkey{qpos}$:]
Show the relative position of
the matching mer. The symbol \texttt{+} in front of the position signifies
a match on the forward strand, while the symbol \texttt{-} signifies
a match on the reverse strand.
\item[$\Showoptionkey{counts}$:]
Show the counts of the
mer, i.e.\ the number of times, the mer occurs in the indexed
sequences.
\item[$\Showoptionkey{sequence}$:]
Show the sequence content of the mer.
\end{description}
For each matching mer, the mentioned values are output on a single line
in the order the four keywords are specified above. Two consecutive values
are separated by white spaces.
}

\Standardoptions
\end{Justshowoptions}

\section{Examples}\label{Examples}

Suppose we have a collection of two files \texttt{read1.fna} and
\texttt{read2.fna}. In the first step, we index
both files using the program \SFX:

\EXECUTE{gt suffixerator -dna -pl -tis -suf -lcp -v -parts 4 -db read1.fna read2.fna -indexname reads}

We get the \SFXidx named \texttt{reads}. Note that we have
used the option \Showoption{parts} with argument 4. This means that
the \SFXidx is created such that only $\frac{1}{4}$th of the
\textsf{suf}-table and the \textsf{lcp}-table of the enhanced suffix array
resided in main memory during the construction. This considerably reduces
the memory requirement.
While this was not really necessary for the small files given in the
index, it is necessary to use this option if the sequence size becomes large.

The created \SFXidx \texttt{reads}
is used in the following call to the program \TYoccratio:

\EXECUTE{gt tallymer occratio -output unique nonunique -minmersize 10 -maxmersize 20 -esa reads}

This shows the counts of $k$-mers for $k\in[10,20]$. The first part of the
output reports counts of unique $k$-mers, while the second is for
non-unique $k$-mers. For example, there are 223755 unique $10$-mers
and 135526 non-unique $10$-mers. If we add the keyword
\Showoptionkey{relative}, then we additionally obtain the fraction
of counts relative to the total number of $k$-mers:

\EXECUTE{gt tallymer occratio -output unique relative -minmersize 10 -maxmersize 20 -esa reads}

For example, we see that $62.3=\frac{223755}{223755+135526}\cdot 100$
percent of all 10-mers are unique. To restrict to specific mer sizes, for
example 10, 13, and 17, we can use option \Showoption{mersizes}:

\EXECUTE{gt tallymer occratio -output unique nonunique -mersizes 10 13 17 -esa reads}

While \TYoccratio can compute distributions for a range of
$k$-mers, \TYmkindex runs for a fixed mer-size, as in the following example:

\EXECUTE{gt tallymer mkindex -mersize 19 -minocc 40 -esa reads}

The output, as explained at the beginning of the output, shows the
distribution of occurrences of 19-mers in the \SFXidx
\texttt{reads}. The 19-mers occurring more
than 40 times are reported with their string content.
We now add options \Showoption{indexname} and \Showoption{counts} to
generate a 19-mer \Tyridx called \texttt{tyr-reads}. The index contains
information to show the counts.

\EXECUTE{gt tallymer mkindex -mersize 19 -minocc 4 -indexname tyr-reads -counts -pl -esa reads}

This generates the 19-mer index file \texttt{tyr-reads.mer} and an additional
table with bucket boundaries stored in file \texttt{tyr-reads.mbd}.

The program \TYsearch now uses the index \texttt{reads} and
matches all 19-mers of the input sequence \texttt{U89959} against it:

\EXECUTE{gt tallymer search -output qseqnum qpos counts sequence -tyr tyr-reads -q U89959.fna | head -n 25}

Each line of the output not beginning with the symbol \texttt{\symbol{35}}
consist of four columns: The first column shows the ordinal number of the
sequence in the query file containing the match. The second number
is the offset in the sequence (counting from 0) whose number is given.
The number is prefixed by the symbol \texttt{+} if the given mer
matches on the forward strand.
The number is prefixed by the symbol \texttt{-} if the given mer
matches on the reverse complemented strand. The third column shows
the occurrence count for the mer in the index. There are separated
counts for the matches on the forward and the reverse complemented strand.
The fourth column shows the mer in forward direction.
%\bibliographystyle{plain}
%\bibliography{defines,kurtz}
\begin{thebibliography}{1}
\bibitem{KUR:NER:STE:WAR:2008}
S.~Kurtz, A.~Narechania, J.C. Stein, and D.~Ware.
\newblock {A new method to compute K-mer frequencies and its application to
  annotate large repetitive plant genomes repetitive plant genomes}.
\newblock {\em {BMC Genomics}}, 9:517, 2008.
\end{thebibliography}

\appendix
\section{Remark to users of previous versions of the programs}
In previous versions of the Tallymer software the programs were 
named differently. The previous program \texttt{vmerstat}
corresponds to \texttt{tallymer mkindex}, \texttt{vmersearch}
corresponds to \texttt{tallymer search}, and \texttt{vmerdist}
corresponds to \texttt{tallymer occratio}. The output format
has not changed in the current version. The options of the
current \Tallymer programs is compatible with the options of the
previous programs. There are only some extra options required:
\begin{enumerate}
\item
the tallymer mkindex-program requires an extra option -esa to
specify the enhanced suffix array. The latter is
constructed by the suffixerator-program which is also part
of the genometools (see examples in Tallymer manual)
\item
seperate runs of constructions of enhanced suffix arrays are no
longer necessary. Instead use the option -parts in one
call to the suffixerator-program which creates an enhanced suffix array
for the entire sequence set. I have tried this for a set of
ESTs (total length 3.2 GB) and it works well.
\item
For very large enhanced suffix arrays (which do not fit into main
memory) use the option -scan for tallymer mkindex and tallymer occratio.
\item 
For tallymer occratio use the option -esa to specify the enhanced suffix
array.
\item
For tallymer search use the option -tyr to specify the input tallymer
index.
\end{enumerate}

\end{document}

\section{TODO-list}
\begin{enumerate}
\item
the current version of \TYmkindex and \TYsearch is optimized for short
mer-sizes (i.e.\ up to length 32 for the DNA alphabet). Longer mer-sizes
lead to large space overheads, and should thus be handled in a different way.
This has to be implemented.
\item
for more than one query file, the program \TYsearch should add the number
of the query file to the output.
\item
add function for selecting statistically relevant mers.
\item
process each query sequence, before inputting the next query sequence.
\begin{comment}
\item
check if all indexes have an alphabet of the same size
\item
add linear scan over mers to construct bucket boundaries.
This is more efficient than binary partitioning in case the
bucket boundaries become small.
\item
change option version such that it does not refer to option Vmatch.
\item
preprocess query sequences by sorting them according to some
prefixlength. Then for the query-substrings beginning with the
same prefix, find the prefix of the given length once and
reuse this information several times.
\end{comment}
\end{enumerate}
