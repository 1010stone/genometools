\documentclass[12pt]{article}
\usepackage[a4paper,top=20mm,bottom=20mm,left=20mm,right=20mm]{geometry}
\usepackage{url}
\usepackage{alltt}
\usepackage{xspace}
\usepackage{times}
\usepackage{listings}
\usepackage{bbm}
\usepackage{verbatim}
\usepackage{prognames}
\usepackage{optionman}
\usepackage{skaff}
\newcommand{\GFM}[0]{\texttt{gfm}\xspace}
\newcommand{\Mup}[1]{\mathit{mup(s,#1)}}
\newcommand{\MS}[1]{\mathit{ms(s,#1)}}
\newcommand{\Lmin}[0]{\mathit{mup(s)}}
\newcommand{\Substring}[3]{#1[#2..#3]}
\newcommand{\EXECUTE}[1]{#1}

\title{\GFM: a program for greedy forward matching\\
a manual}
\author{\begin{tabular}{c}
         \textit{Stefan Kurtz}\\
         Center for Bioinformatics,\\
         University of Hamburg
        \end{tabular}}

\begin{document}
\maketitle

\section{The program \GFM}

The program \GFM is called as follows:
\par
\noindent\GFM [\textit{options}] \Showoption{query} \Showoptionkey{queryfile}
\par
The given \textit{queryfile} must be in multiple 
\Fasta format and can optionally be gzipped, in which case it must end 
with the suffix \texttt{gz}.
Each sequence in \texttt{queryfile} is called \textit{unit} in the following.
In addition to the mandatory option \Showoption{query}, the program
must be called with either option \Showoption{pck} or \Showoption{esa}
which specify to use a packed index or an enhanced suffix array for 
a given set of subject sequences.

The program runs in basically two modes. 

\begin{enumerate}
\item
In the default mode \GFM computes
for all positions \(i\) in each unit, say \(s\) of length
\(n\), the length \(\Mup{i}\) of the minimum unique prefix 
at position \(i\), if it exists. Uniqueness always refers to all substrings
represented by the index. \(\Mup{i}\) is defined by the following two 
statements:
\begin{itemize}
\item
If \(\Substring{s}{i}{n-1}\) is not unique in the index, then \(\Mup{i}=\bot\).
That is, it is undefined.
\item
If \(\Substring{s}{i}{n-1}\) is unique in the index, then \(\Mup{i}=m\), where 
\(m\) is the smallest value such that \(i+m-1\leq n-1\) and 
\(\Substring{s}{i}{i+m-1}\) occurs exactly once as a substring in the index.
\end{itemize}
Note that it is possible that for all \(i\in[0,n-1]\) we have 
\(\Mup{i}=\bot\), which means that unit \(s\) does not contain any unique 
substring. In this case, the program reports nothing for the corresponding
unit.
\item
In the other mode, i.e.\ when option \Showoption{ms} is used, the 
\textit{matching statistics} is computed. That is, for each position \(i\) in 
each unit, say \(s\) of length \(n\), \(\MS{i}=(l,j)\) is computed. Here
\(l\) is the largest integer such that \(\Substring{s}{i}{i+l-1}\) matches
a substring represented by the index and \(j\) is a start position of the
matching substring in the index. We say that \(l\) is the length of \(\MS{i}\)
and \(j\) is the subject position of \(\MS{i}\).
\end{enumerate}

The following options are available in \GFM:

\begin{Justshowoptions}
\Option{ms}{}{
Compute matching statistics of the units in the query file against the
given index.
}

\Option{fmi}{$\Showoptionarg{indexname}$}{
Use the old implementation of the FMindex. This option is not recommended.
}

\Option{esa}{$\Showoptionarg{indexname}$}{
Use the given enhanced suffix array to compute the matches.
}

\Option{pck}{$\Showoptionarg{indexname}$}{
Use the packed index (an efficient representation of the FMindex)
to compute the matches.
}

\Option{query}{$\Showoptionarg{files}$}{
Specify a white space separated list of query files containing the units.
At least one query file must be given. The files may be in 
gzipped format, in which case they have to end with the suffix \texttt{.gz}.
}

\Option{min}{$\ell$}{
Specify the minimum length $\ell$ of the matches minimum unique prefixes or 
matching statistics length. That is, in the default case,
for each unit \(s\) and each positions \(i\) in \(s\), the program reports
the values \(i\) and \(\Mup{i}\) whenever \(\Mup{i}\geq\ell\). 
When using option \Showoption{ms}, for each unit \(s\) and each positions 
\(i\) in \(s\), the program reports all values \(i\) and \(\MS{i}\) if the 
length of \(\MS{i}\) is at lest \(\ell\).
}

\Option{max}{$\ell$}{
Specify the maximum length $\ell$ of minimum unique prefixes or matching
statistics length.  That is, in the default case,
for each unit \(s\) and each positions \(i\) in \(s\), the program reports the
values \(i\) and \(\Mup{i}\) whenever \(\Mup{i}\leq\ell\). When using
option \Showoption{ms}, for each unit \(s\) and each positions \(i\) in
\(s\), the program reports all values \(i\) and \(\MS{i}\) if the length
of \(\MS{i}\) is at most \(\ell\).
}

\Option{output}{(\Showoptionkey{subjectpos}$\mid$\Showoptionkey{querypos}$\mid$\Showoptionkey{sequence})}{
Specify what to output. At least one of the three keys words
$\Showoptionkey{subjectpos}$,
$\Showoptionkey{querypos}$, and
$\Showoptionkey{sequence}$ must be used. The keyword
$\Showoptionkey{subjectpos}$ is only possible in combination with option
\Showoption{ms}. Using the keyword $\Showoptionkey{subjectpos}$ shows the 
subject position of the matching statistics.
Using the keyword $\Showoptionkey{querypos}$ shows the query position.
Using the keyword $\Showoptionkey{sequence}$ shows the sequence content
of the match.
}

\Helpoption

\end{Justshowoptions}
The following conditions must be satisfied:
\begin{enumerate}
\item
If both options \Showoption{min} and \Showoption{max} are used, then
the value specified by option \(\Showoption{min}\) must be smaller
than the value specified by option \(\Showoption{max}\).
\end{enumerate}

\section{Examples}

Suppose that in some directory, say \texttt{home-sapiens}, we have 25 gzipped
fasta files containing 24 human chromomsomes plus one file with mitrochondrial
sequences.  In the first step, we construct the index for these files:

\begin{Output}
gt packedindex mkindex -dna -dir rev -parts 15 -bsize 10 -locfreq 32
                       -indexname human-all -db home-sapiens/*.gz
\end{Output}

The program runs for almost two hours and delivers 
an index \texttt{human-all} consisting of three files:

\begin{Output}
ls -lh human-all.*
-rw-r----- 1 kurtz gistaff   37 2008-01-24 00:47 human-all.al1
-rw-r----- 1 kurtz gistaff 1.9G 2008-01-24 02:37 human-all.bdx
-rw-r----- 1 kurtz gistaff 3.4K 2008-01-24 02:37 human-all.prj
\end{Output}

This is used in the following call to the program \GFM:

\begin{Output}
gfm -output querypos -min 20 -max 30 -query queryfile.fna -pck human-all
unit 0 (Arabidopsis thaliana BAC T7I23, complete sequence)
1007 20
1010 22
1011 22
1012 21
1013 21
...
\end{Output}

For all units \(s\) in the multiple \Fasta file \texttt{queryfile.fna},
a line is shown, reporting the number of the unit and the original fasta
header. Also, all positions \(i\) in \(s\) with \(\Mup{i}=\Lmin\) are reported.
The first column is the relative position in the query sequence (counting 
from 0). The second column shows the value \(\Mup{i}\) whenever
\(i\) is the value in the first column.

To additionally report the sequence content of the
minimum unique prefixes we add the keyword \Showoption{sequence} to option
\Showoption{-output}:

\begin{Output}
gfm -output querypos sequence -min 20 -max 30 -query queryfile.fna -pck human-all
unit 0 (Arabidopsis thaliana BAC T7I23, complete sequence)
unit 0 (Arabidopsis thaliana BAC T7I23, complete sequence)
1007 20 ctgacagtttttttttttta
1010 22 acagttttttttttttacttta
1011 22 cagttttttttttttactttat
1012 21 agttttttttttttactttat
1013 21 gttttttttttttactttata
...
\end{Output}

Next we compute the matching statistics by using the option \Showoption{ms}.
unit 0 (Arabidopsis thaliana BAC T7I23, complete sequence)

\begin{Output}
gfm -ms -output subjectpos querypos sequence -min 20 -max 30
    -query queryfile.fna -pck human-all 
22 20 390765125 actgtatctcaaaatataaa
253 21 258488266 gggaataaacatgtcattgag
254 20 258488267 ggaataaacatgtcattgag
275 20 900483549 taattctatttttctttctt
480 20 1008274536 gcttgaagatcatgatccag
...
\end{Output}

Here the first column is the relative position \(i\) in the unit. The second
column shows the value \(l\) and the third the value \(j\), where
\(\MS{i}=(l,j)\) the third reports the matching sequence.

\end{document}
